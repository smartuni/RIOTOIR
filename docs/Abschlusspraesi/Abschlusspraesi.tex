\documentclass{beamer}
\usepackage[utf8]{inputenc}
\usepackage[ngerman]{babel}
\usetheme{JuanLesPins}  %% Themenwahl

\usepackage[orientation=portrait,size=a0,scale=1.4,debug]{beamerposter}

\usepackage{pgfpages}
\setbeamertemplate{note page}[plain]
%\setbeameroption{show notes on second screen=right}

\title{RIOTOIR}
\author{Martin Witte}
\date{\today}

\begin{document}
%\beamertemplatenavigationsymbolsempty
\unitlength 5mm


\begin{frame} %%Eine Folie
  \begin{block}{\textbf{Überblick}}
  	  \centering
	  \includegraphics[width=\linewidth]{riotoir-logo.png}
	  \begin{columns}[t]
	  	\begin{column}{.50\linewidth}
		  \begin{block}{IR-Übertragung im RIOT-Netzwerkstack}
		    \begin{align*}
		    &RIOT &&\Rightarrow~~~~Betriebssystem~f\ddot{u}r~IoT-Ger\ddot{a}te\\
		    &O &&\Rightarrow~~~~\textbf{O}n\\
		    &I &&\Rightarrow~~~~\textbf{I}nfra\\
		    &R &&\Rightarrow~~~~\textbf{R}ed\\
		    \end{align*}
		  \end{block}
	  \end{column}
  \end{columns}
	\end{block}
\vfill
  \begin{block}{\textbf{Details}}
    \begin{columns}[t]
      \begin{column}{.7\linewidth}
		\begin{block}{Übertragung}
			- Baudrate (des ersten Prototyp) $\rightarrow$ $1333 Bd - 1838 Bd$\\
			- Eigene Modulation (auf geringe Timergenauigkeit ausgelegt)\\
			- Uni- oder Bidirektional möglich\\
			- Halbduplex möglich\\
		\end{block}
	  \end{column}
    \end{columns}
\vspace{10pt}
	\begin{columns}[t]
	  \begin{column}{.45\linewidth}
		\begin{block}{Sender}
			Software:\\
			- Errechnet aus den zu sendenden Daten eine Sequenz von Spannungsänderungen\{0V,3.3V\} an einem Pin des Mikrokontrollers, welche die Senderperipherie ansteuert.\\
			Hardware:\\
			- Die Spannung vom Pin des Mikrokontrollers schaltet den Transistor durch, wodurch die IR-Diode mit Energie zum "leuchten" versorgt wird.\\
			- Aufbau mit nur 2 Bauteilen(IR-Diode, Transistor)\\
			~\\
		\end{block}
	  \end{column}
      \begin{column}{0.45\linewidth}
		\begin{block}{Empfänger}
			Software:\\
			- Wenn die Empfänger-Peripherie die Spannung am Pin ändert, wird ein Interrupt ausgelöst. Dieser stößt dann die weitere Verarbeitung des Signals an.\\
			- Realisiert mit Interrupts und Message-Passing\\
			Hardware:\\
			- Die eintreffende IR-Strahlung schaltet den IR-Transistor durch, dieser wiederum schaltet einen PNP-Transistor durch. Dadurch wird eine Spannung\{0V, 3.3V\} an einen Pin des Mikroprozessors gelegt\\
			- Aufbau mit nur 4 Bauteilen(IR-Transistor, Transistor, Widerstände)\\
		\end{block}
	  \end{column}
	\end{columns}
	\vspace{10pt}
  \end{block}
\end{frame}

\end{document}
